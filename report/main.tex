\documentclass[11pt]{article}

\usepackage[T1]{fontenc}
\usepackage{babel}
\usepackage[a4paper, total={6in, 10in}]{geometry}

\usepackage{float}
\usepackage[usenames,dvipsnames]{xcolor}
\usepackage[most]{tcolorbox}
\usepackage{listings} % per scrittura di codice
\usepackage{algorithmic}
\usepackage{algorithm}
\usepackage{booktabs}
\usepackage{multirow}
\usepackage{hyperref}
\hypersetup{
    colorlinks=true,
    linkcolor=blue,
    filecolor=cyan,      
    urlcolor=magenta,
    pdftitle={Overleaf Example},
    pdfpagemode=FullScreen,
    }


\usepackage{graphicx}
\graphicspath{{./images}}

\definecolor{InlineGray}{HTML}{E3E3E3}

% this style should be active for all lstlistings environments
\lstdefinestyle{all}{
 % columns=flexible,
 tabsize=2,
 columns=fixed,
 keepspaces=true,
 frameshape={}{y}{}{}, % per rimuovere la linea vericale sulla sinistra `frameshape={}{}{}{}`
 basicstyle = {\ttfamily \color{black}},
 showstringspaces=false,
 rulecolor=\color{black},
 numberstyle=\tiny\color{lightgray}
}

\lstdefinestyle{inline}{
 basicstyle={\small\ttfamily \color{black}},
 columns=fixed,
 showstringspaces=false,
}

\lstdefinestyle{SQLStyle}
{
 language=SQL,
 breaklines=true,
 breakatwhitespace=true,
 stringstyle=\color{red},
 commentstyle=\color{gray},
 keywordstyle=\color{violet},
 keywordstyle=[2]\color{cyan},
 keywordstyle=[3]\color{violet},
 morekeywords=[2]{explode,split,length,row_number,array_intersect, array_union, sort_array},
 morekeywords=[3]{PARTITION,LATERAL}
}


\lstdefinestyle{PythonStyle}
{
 language=Python,
 breaklines=true,
 breakatwhitespace=true,
 stringstyle=\color{Bittersweet},
 commentstyle=\color{darkgray},
 keywordstyle=\color{violet},
 keywordstyle=[2]\color{violet},
 morekeywords=[2]{agg,avg,alias,sort,ascending,groupBy,withColumn,filter,join,withColumnRenamed,where,select,desc,drop,row_number,over}
}

\tcbset{on line, 
    boxsep=2pt, left=0pt,right=0pt,top=0pt,bottom=0pt,
    colframe=white,colback=InlineGray,
}
\newcommand{\codeinline}[1]{\tcbox{\lstinline[style=inline]|#1|}}

\author{
 Stefanelli Francesco \\
 Matricola 538549\\
 \texttt{fra.stefanelli3@stud.uniroma3.it}
 \and
 Galletti, Davide \\
 Matricola 533152\\
 \texttt{dav.galletti@stud.uniroma3.it}
}

\title{\huge\textbf{Report Primo Progetto Big Data}}
\date{\textbf{Data Team} \\
\href{https://github.com/BD-Data-Team/first-project-bigdata}{GitHub: data-team}\\
Anno Accademico 2022/2023}



\begin{document}
\maketitle
In questa relazione è stato documentato il lavoro svolto durante lo sviluppo del primo progetto del corso di Big data. La relazione di divide in tre sezioni: nella prima sezione è riportata la fase di preparazione dei dati, nella seconda è riportata l'implementazione dei tre job e nella terza si presentano i risultati e le prestazioni ottenute su cluster ed in locale.

\section{Data preparation}
Per la fase di data preparation si è scelto di utilizzare un \textbf{Notebook Jupyter} insieme a \textbf{Python} utilizzando \textbf{Pandas}.
Sono state applicate \codeinline{dropna()} e \codeinline{drop_duplicates()} pre- e post-processing per cancellare le ennuple del dataset contenenti valori nulli e duplicati e inoltre abbiamo pulito le colonne del dataset testuali (Text, Summary, ProfileName) convertendo il loro contenuto a lowercase, rimosso i caratteri non alfanumerici, spazi multipli e effettuata una \codeinline{strip()}.


\section{Implementazione}\label{section:1}
In questa sezione sono riportate le implementazioni dei vari job nelle tecnologie di \textbf{Map Reduce}, \textbf{Hive}, \textbf{Spark Core} e \textbf{Spark SQL}.

\subsection{Job 1}
\textit{Un job che sia in grado di generare, per ciascun anno, i 10 prodotti che hanno ricevuto il maggior numero di recensioni e, per ciascuno di essi, le 5 parole con almeno 4 caratteri più frequentemente usate nelle recensioni (campo text), indicando, per ogni parola, il numero di occorrenze della parola.}
  \subsubsection{Map Reduce}
  Per l'implementazione del primo job in map reduce si è pensato di suddividere l’elaborazione in due passate map reduce, dove:
  \begin{itemize}
    \item \textbf{Prima passata}: si calcolano i top 10 prodotti per ogni anno.
    
    \begin{algorithm}[!ht]
    \caption{Mapper1}
      \begin{algorithmic}[1]
        \STATE \textbf{Input}: CSV data from STDIN
        \STATE \textbf{Output}: year, product\_id and text of the review
        \STATE Initialize $cols$ with dataset column names
  
        \FORALL{line in STDIN}
        \STATE row $\gets$ parsed data from line based on $cols$
        
        \STATE skip CSV header
    
        \STATE $year \gets$ row[Year]
        \STATE $product\_id \gets$ row[ProductId]
        \STATE $text \gets$ row[Text]
        
        \STATE \textbf{print}(\{$year$\}\texttt{\textbackslash t}\{$product\_id$\}\texttt{\textbackslash t}\{$text$\})
       \ENDFOR
      \end{algorithmic}
    \end{algorithm}

    \begin{algorithm}[!ht]
      \caption{Reducer1}
      \begin{algorithmic}[1]
      \STATE \textbf{Input}: input data from STDIN
      \STATE \textbf{Output}: year, product\_id and text of the top 10 product for each year
      \STATE Initialize $year\_for\_product\_2\_sum$ as a Counter Object
      \STATE Initialize $year\_for\_product\_2\_text$ as a Dictionary
      \FORALL{$line$ in STDIN}
        \STATE $year$, $product\_id$, $text$ $\gets$ line.split("\texttt{\textbackslash t}")
        \STATE $year\_for\_product\_2\_sum$[$year$][$product\_id$]$++$
        \STATE $year\_for\_product\_2\_text$[$year$][$product\_id$].append($text$)
      \ENDFOR
      
      \FORALL{$year$ in $year\_for\_product\_2\_sum$}
        \STATE $top\_10\_products \gets$ top 10 in $year\_for\_product\_2\_sum$[$year$]
        \FORALL{$product$ in $top\_10\_products$}
          \FORALL{$text$ in $year\_for\_product\_2\_text$[$year$][$product$]}
            \STATE \textbf{print} (\{$year$\}\texttt{\textbackslash t}\{$product$\}\texttt{\textbackslash t}\{$text$\})
          \ENDFOR
        \ENDFOR
      \ENDFOR
    \end{algorithmic}
    \end{algorithm}
\newpage
    \item \textbf{Seconda passata}: si calcolano le 5 parole più frequenti per ciascuno dei top 10 prodotti di ogni anno.
    
    \begin{algorithm}[!ht]
      \caption{Mapper2}
      \begin{algorithmic}[1]
      \STATE \textbf{Input}: input data from STDIN
      \STATE \textbf{Output}: year, product\_id and the words with their relative count of the top 10 products for each year
      
      \FORALL{$line$ in STDIN}
        \STATE $year$, $product\_id$, $text$ $\gets$ line.split("\texttt{\textbackslash t}")
      
        \STATE $words$ $\gets$ $text$.split(" ")

        \STATE $counted\_words$ $\gets$ Dictionary of occurrences from $words$
      
        \FORALL{$word$, $count$ in $counted\_words$}
          \IF{length($word$) $\geq$ 4}
            \STATE \textbf{print} (\{$word$\}\texttt{\textbackslash t}\{$year$\}\texttt{\textbackslash t}\{$product\_id$\}\texttt{\textbackslash t}\{$count$\}"
          \ENDIF
        \ENDFOR
      \ENDFOR
    \end{algorithmic}
    
    \end{algorithm}
    \begin{algorithm}[!ht]
    \caption{Reducer2}
    \begin{algorithmic}[1]
    \STATE \textbf{Input}: input data from STDIN
    \STATE \textbf{Output}: year, product\_id, words and their count of the 5 most common words of the top 10 products for each year 
    \STATE Initialize $year\_for\_word\_2\_count$ as a Dictionary
    \FORALL{line in STDIN}
        \STATE $word$, $year$, $product\_id$, $count$ $\gets$ line.split("\texttt{\textbackslash t}")
    
    
        \STATE $year\_for\_word\_2\_count$[$year$][$product\_id$][$word$] $\mathrel{+}= text{count}$
    \ENDFOR
    \FORALL{$year$ in $year\_for\_word\_2\_count$}
        \FORALL{$product\_id$ in $year\_for\_word\_2\_count$[$year$]}
            \STATE top\_5\_words $\gets$ top 5 words in $year\_for\_word\_2\_count$[$year$][$product\_id$]
            \FORALL{$word$, $count$ in }
                \STATE \textbf{print} (\{$year$\}\texttt{\textbackslash t}\{$product\_id$\}\texttt{\textbackslash t}\{$word$\}\texttt{\textbackslash t}\{$count$\})
            \ENDFOR
        \ENDFOR
    \ENDFOR
    \end{algorithmic}
    \end{algorithm}  
  \end{itemize}
  
\newpage
  \subsubsection{Hive}
  Per l'implementazione in Hive si è pensato di suddividere il job in varie query intermedie che facessero apprezzare meglio l'interrogazione completa, man mano salvate in tabelle. Le TABLE intermedie sono illustrate di seguito:
  \begin{itemize}
    \item \codeinline{reviews\_per\_year}: numero di recensioni di ciascun prodotto all'interno dell'anno.
    \begin{lstlisting}[style = all, style = SQLStyle]
CREATE TABLE if not exists reviews_per_year AS
SELECT review_year, product_id, collect_list(text) as texts, COUNT(*) as reviews_count
FROM reviews
GROUP BY review_year, product_id;
    \end{lstlisting}
    \item \codeinline{top\_10\_products\_for\_year}: top 10 dei prodotti più recensiti per ciascun anno.
    \begin{lstlisting}[style = all, style = SQLStyle]
CREATE TABLE if not exists top_10_products_for_year AS
SELECT review_year, product_id, texts
FROM (
        SELECT *, row_number() OVER (PARTITION BY review_year ORDER BY reviews_count DESC) as row_num
        FROM reviews_per_year 
    ) as ranked_reviews_per_year
WHERE row_num <= 10; 
    \end{lstlisting}
    
    \item \codeinline{year\_for\_product\_2\_word\_count}: calcolo delle occorrenze di ciascuna parola (con almeno 4 caratteri) di ciascun prodotto all'interno dell'anno.
    \begin{lstlisting}[style=all, style = SQLStyle]
CREATE TABLE if not exists year_for_product_2_word_count AS
SELECT review_year, product_id, exploded_text.word as word, COUNT(*) as word_count
FROM (
  SELECT review_year, product_id, exploded_texts.text
  FROM top_10_products_for_year
  LATERAL VIEW explode(texts) exploded_texts AS text
) AS t
LATERAL VIEW explode(split(text, ' ')) exploded_text AS word
WHERE length(exploded_text.word) >= 4
GROUP BY review_year, product_id, exploded_text.word;
    \end{lstlisting}
    
    \item \codeinline{output query}: top 10 dei prodotti più recensiti per ciascun anno con le 5 parole (con almeno 4 caratteri) più frequentemente usate.
    \begin{lstlisting}[style=all, style = SQLStyle]
SELECT review_year, product_id, word, word_count
FROM (  SELECT *, row_number() OVER (PARTITION BY review_year, product_id ORDER BY word_count DESC) as row_num
        FROM year_for_product_2_word_count 
    ) as ranked_year_for_product_2_word_count
WHERE row_num <= 5;
    \end{lstlisting}
  \end{itemize}
  
  
  \subsubsection{Spark Core}
  \begin{algorithm}
    \caption{Spark Application}
    \begin{algorithmic}[1]
    
    \STATE \textbf{Input}: input dataset
    \STATE \textbf{Output}: text file containing the 5 most common words for each of the 10 most common products of each year from the input dataset
    
    \STATE Initialize $input\_rdd$ from the input dataset
    \STATE Filter out the header line from the $input\_rdd$
    
    \STATE $year\_and\_product\_rdd$ $\gets$ Map ($year$,$product\_id$) from $input\_rdd$
    \STATE $productId\_and\_text\_rdd$ $\gets$ Map (($year$,$product\_id$),$text$) from $input\_rdd$
    \STATE Group $year\_and\_product\_rdd$ by $year$
    \STATE $top\_10\_for\_each\_year\_rdd$ $\gets$ Map and flatten $year\_and\_product\_rdd$ to get (year, product) pairs for the 10 most reviewed products for each year
    
    \STATE $top\_10\_for\_each\_year\_with\_reviews\_rdd$ $\gets$ Join the $top\_10\_for\_each\_year\_rdd$ with $productId\_and\_text\_rdd$
    
    \STATE Group the $top\_10\_for\_each\_year\_with\_reviews\_rdd$ by year and productId
    
    \STATE Define a function to get the most common words for each (year, product) pair
    
    \STATE $output\_rdd$ $\gets$ Map the $top\_10\_for\_each\_year\_with\_reviews\_rdd$ to get the most common words for each (year, product) pair
    \STATE Save the $output\_rdd$ as text files
    
    \end{algorithmic}
    \end{algorithm}

  \subsubsection{Spark SQL}
  L'implementazione del primo job in spark-sql è stata divisa nelle seguenti operazioni:
  \begin{itemize}
    \item \codeinline{reviews\_per\_year\_DF}: numero di recensioni di ciascun prodotto all'interno dell'anno.
  \begin{lstlisting}[style=all, style=PythonStyle]
reviews_per_year_DF = input_DF.groupBy( "review_year", "product_id").count().withColumnRenamed("count", "reviews_count")
  \end{lstlisting}
    \item \codeinline{top\_10\_products\_for\_year\_DF}: top 10 dei prodotti più recensiti per ciascun anno.
  \begin{lstlisting}[style=all, style=PythonStyle]
win_temp = Window.partitionBy("review_year").orderBy(
    reviews_per_year_DF["reviews_count"].desc())
top_10_products_for_year_DF = reviews_per_year_DF.withColumn("row_num", row_number().over(win_temp)).filter("row_num <= 10").drop("row_num").drop("reviews_count")
\end{lstlisting}
  \item \codeinline{top\_10\_products\_for_year\_with\_reviews\_DF}: top 10 dei prodotti per ciascun anno con le recensioni.
     \begin{lstlisting}[style=all, style=PythonStyle]
join_condition = [top_10_products_for_year_DF.review_year == input_DF.review_year, top_10_products_for_year_DF.product_id == input_DF.product_id]
top_10_products_for_year_with_reviews_DF = top_10_products_for_year_DF.join(input_DF, join_condition).drop(input_DF.product_id, input_DF.review_year).select("review_year", "product_id", "text")
\end{lstlisting}

    \item \codeinline{year\_for\_product\_2_word\_count\_DF}: calcolo delle occorrenze di ciascuna parola (con almeno 4 caratteri) di ciascun prodotto all'interno dell'anno.
     \begin{lstlisting}[style=all, style=PythonStyle]
top_10_products_for_year_with_reviews_DF = top_10_products_for_year_with_reviews_DF.withColumn("word", explode(split(top_10_products_for_year_with_reviews_DF.text, " ")))
year_for_product_2_word_count_DF = top_10_products_for_year_with_reviews_DF.groupBy( "review_year", "product_id", "word").count().withColumnRenamed("count", "word_count")\ .where("length(word) >= 4").select("review_year", "product_id", "word", "word_count")
\end{lstlisting}

    \item \codeinline{output\_DF}: top 10 dei prodotti più recensiti per ciascun anno con le 5 parole (con almeno 4 caratteri) più frequentemente usate.
     \begin{lstlisting}[style=all, style=PythonStyle]
win_temp2 = Window.partitionBy("review_year", "product_id").orderBy( year_for_product_2_word_count_DF["word_count"].desc())
output_DF = year_for_product_2_word_count_DF.withColumn("row_num", row_number().over(win_temp2)).filter("row_num <= 5").drop("row_num").drop("reviews_count")
\end{lstlisting}
  \end{itemize}


\newpage
\subsection{Job 2}
\textit{Un job che sia in grado di generare una lista di utenti ordinata sulla base del loro apprezzamento, dove l’apprezzamento di ogni utente è ottenuto dalla media dell’utilità (rapporto tra HelpfulnessNumerator e HelpfulnessDenominator) delle recensioni che hanno scritto, indicando per ogni utente il loro apprezzamento.}

  \subsubsection{Map Reduce}
    Per l'implementazione del primo job in map reduce si è pensato di suddividere l'eleborazione in due passate map reduce, dove:
  \begin{itemize}
    \item \textbf{Prima passata}: si effettua l'estrazione dei valori di User Id, Helpfulness Numerator e Denominator per il calcolo dell'Apprezzamento (Appreciation) di ogni utente;
    \begin{algorithm}[!ht]
    \caption{Mapper1}
      \begin{algorithmic}[1]
        \STATE \textbf{Input}: CSV data from STDIN
        \STATE \textbf{Output}: user\_id and usefulness
        \STATE Initialize $cols$ with dataset column names
  
        \FORALL{line in STDIN}
        \STATE row $\gets$ parsed data from line based on $cols$
        
        \STATE skip CSV header
    
        \STATE $user\_id \gets$ row[UserId]
        \STATE $helpfulness\_num \gets$ row[HelpfulnessNumerator]
        \STATE $helpfulness\_den \gets$ row[HelpfulnessDenominator]
        
        \IF{$helpfulness\_num > helpfulness\_den \,\|\, helpfulness\_den \leq 0$}
         \STATE \textbf{continue}
        \ENDIF
      
        \STATE $usefulness \gets helpfulness\_num \,/\, helpfulness\_den$
        
        \STATE \textbf{print}(\{$user\_id$\}\texttt{\textbackslash t}\{$usefulness$\})
       \ENDFOR
      \end{algorithmic}
    \end{algorithm}
    \begin{algorithm}[!ht]
      \caption{Reducer1}
      \begin{algorithmic}[1]
        \STATE \textbf{Input}: input data from STDIN
        \STATE \textbf{Output}: user and appreciation
        \STATE Initialize $user\_2\_count$ as a Counter object
        \STATE Initialize $user\_2\_usefulness$ as a Counter object
        
        \FORALL{line in STDIN}
          \STATE $id,\, usefulness \gets$ split line using tab character \texttt{\textbackslash t}
            \STATE Increment $user\_2\_count[id]$ by 1
            \STATE Increment $user\_2\_usefulness[id]$ by $curr\_usefulness$
        \ENDFOR
        
        \FORALL{$user$ in $user\_2\_count$}
          \STATE $appreciation \gets user\_2\_usefulness[user] \,/\, user\_2\_count[user]$
          \STATE \textbf{print}(\{$user$\}\texttt{\textbackslash t}\{$appreciation$\})
        \ENDFOR
      \end{algorithmic}
    \end{algorithm}
    \item \textbf{Seconda passata}: si effettua il sort dell'output della prima passata sulla base del valore di Apprezzamento (Appreciation) del singolo utente.
    \begin{algorithm}[!ht]
    \caption{Mapper2}
    \begin{algorithmic}[1]
    \STATE \textbf{Input}: input data from STDIN
    \STATE \textbf{Output}: appreciation and user
    \FORALL{line in input data}
      \STATE user, appreciation $\gets$ line.split("\texttt{\textbackslash t}")
      \STATE \textbf{print} \{$appreciation$\}\texttt{\textbackslash t}\{$user$\}
    \ENDFOR
  \end{algorithmic}
  \end{algorithm}

  \begin{algorithm}
    \caption{Reducer2}
    \begin{algorithmic}[1]
    \STATE \textbf{Input}: input data from STDIN
    \STATE \textbf{Output}: user and appreciation
    
    \FORALL{line in input data}
      \STATE line $\gets$ line.strip()
      \STATE appreciation, user $\gets$ line.split("\texttt{\textbackslash t}")
      \STATE \textbf{print} \{$user$\}\texttt{\textbackslash t}\{$appreciation$\}
    \ENDFOR
    
    \end{algorithmic}
  \end{algorithm}
  \end{itemize}

\newpage
  \subsubsection{Hive}
  Per l'implementazione del secondo job in hive è stata effettua tutta l'operazione di calcolo in una singola query utilizzando la funzione di media.

\begin{lstlisting}[style = all, style = SQLStyle]
SELECT user_id, AVG(helpfulness_numerator / helpfulness_denominator) as appreciation
FROM reviews
WHERE NOT (helpfulness_numerator > helpfulness_denominator OR helpfulness_denominator <= 0.0)
GROUP BY user_id
ORDER BY appreciation DESC;
\end{lstlisting}
  
  
  \subsubsection{Spark Core}
  Per l'implementazione del secondo job in spark-core sono state applicate tre funzioni di \codeinline{map()}, una \codeinline{reduceByKey()}, due funzioni \codeinline{filter()} ed una \codeinline{sortBy()}.
  \begin{algorithm}
    \caption{Spark Application}
    \begin{algorithmic}[1]
        \STATE \textbf{Input}: input dataset
        \STATE \textbf{Output}: a text file containing $user\_ids$ and their relative $appreciation$, sorted by value
        
        \STATE Define $transform\_data()$: a function that extracts the helpfulnessNumerator and helpfulnessDenominator from a dataset line and compute its division
        \STATE Initialize $input\_rdd$ from the input dataset
        \STATE Filter out the header line from the $input\_rdd$

        \STATE $helpfulness\_rdd$ $\gets$ Map ($user\_id$, $helpfulness$) from $input\_rdd$ using $transform\_data()$
        \STATE Filter out $helpfulness > 1.0$ from $helpfulness\_rdd$

        \STATE $helpfulness\_rdd$ $\gets$ Map ($user\_id$, ($helpfulness$, 1)) from $helpfulness\_rdd$
        
        \STATE $helpfulness\_rdd$ $\gets$ Reduce By Key ($user\_id$, (sum of $helpfulness$,  sum of 1)) from $helpfulness\_rdd$
        
        \STATE $appreciation\_rdd$ $\gets$ Map ($user\_id$, $helpfulness\_sum\,/\, count$) from $helpfulness\_rdd$
        
        \STATE $appreciation\_rdd$ $\gets$ Sort By Value $appreciation\_rdd$

        \STATE Save as text file $appreciation\_rdd$
    \end{algorithmic}
    \end{algorithm}
    
  
  \subsubsection{Spark SQL}
  Per l'implementazione del secondo job in spark-sql è stata divisa in due operazioni:
  \begin{itemize}
    \item \textbf{Prima operazione}: Calcolo dell'\codeinline{appreciation} per ciascun \codeinline{product\_id} partendo dal dataset di input per mezzo di una proiezione ed una operazione di media.
  \begin{lstlisting}[style=all, style=PythonStyle]
# df = dataframe containing the input dataset
df = df.withColumn("Helpfulness", df["HelpfulnessNumerator"] / df["HelpfulnessDenominator"]) \
.groupBy("UserId").agg(F.avg("Helpfulness").alias("Appreciation"))
  \end{lstlisting}
    \item \textbf{Seconda operazione}: Filtraggio e Sort su \codeinline{Appreciation}
  \begin{lstlisting}[style=all, style=PythonStyle]
df = df.filter(df["Appreciation"] <= 1.0) \
    .sort("Appreciation", ascending=False)
  \end{lstlisting}
  \end{itemize}
  
\subsection{Job 3}
\textit{Un job in grado di generare gruppi di utenti con gusti affini, dove gli utenti hanno gusti affini se hanno recensito con score superiore o uguale a 4 almeno 3 prodotti in comune, indicando, per ciascun gruppo, i prodotti condivisi. Il risultato deve essere ordinato in base allo UserId del primo elemento del gruppo e non devono essere presenti duplicati.}
  \subsubsection{Map Reduce}
  Per l'implementazione del primo job in map reduce si è pensato di suddividere l'eleborazione in quattro passate map reduce, dove:
  \begin{itemize}
      \item \textbf{Prima passata}: si effettua il la rimozione delle \codeinline{reviews} che non presentano uno score di almeno \codeinline{4} e si creano delle liste di utenti per ciascun prodotto filtrando i prodotti che non sono stati comprati da almeno due utenti 

      \begin{algorithm}[!ht]
    \caption{Mapper1}
      \begin{algorithmic}[1]
        \STATE \textbf{Input}: CSV data from STDIN
        \STATE \textbf{Output}: product\_id and user\_id of the reviews with a score >=
        \STATE Initialize $cols$ with dataset column names
  
        \FORALL{line in STDIN}
        \STATE row $\gets$ parsed data from line based on $cols$
        
        \STATE skip CSV header
    
        \STATE $user\_id \gets$ row[UserId]
        \STATE $product\_id \gets$ row[ProductId]
        \STATE $score \gets$ row[Score]

        \IF{$score >= 4$}
            \STATE \textbf{print}(\{$product\_id$\}\texttt{\textbackslash t}\{$user\_id$\}) 
        \ENDIF
       \ENDFOR
      \end{algorithmic}
    \end{algorithm}

    \begin{algorithm}[!ht]
      \caption{Reducer1}
      \begin{algorithmic}[1]
      \STATE \textbf{Input}: input data from STDIN
      \STATE \textbf{Output}: product\_id and the list of users who have bought this product\_id 
      \STATE Initialize $product\_2\_users$ as a Dictionary of Sets
      \FORALL{$line$ in STDIN}
        \STATE $product\_id$, $user\_id$ $\gets$ line.split("\texttt{\textbackslash t}")
        \STATE $product\_2\_users$[$product\_id$].append($user\_id$)
      \ENDFOR
      
      \FORALL{$product\_id$ in $product\_2\_users$}
        \IF{\textbf{size}($product\_2\_users$[$product\_id$]) $>= 2$}
        \STATE \textbf{print}(\{$product\_id$\}\texttt{\textbackslash t}\{$product\_2\_users$[$product\_id$]\})
        \ENDIF
      \ENDFOR
    \end{algorithmic}
    \end{algorithm}
    
\newpage

    \item \textbf{Seconda e terza passata}: Questo step viene eseguito due volte di fila e si calcola l'insieme di utenti che hanno comprato un determinato insieme di prodotti. (Applicaldo il ragionamento per 2 volte si ottengono liste di 3/4 prodotti e tutti gli utenti che li hanno comprati.)

      \begin{algorithm}[!ht]
    \caption{Mapper2}
      \begin{algorithmic}[1]
        \STATE \textbf{Input}: CSV data from STDIN
        \STATE \textbf{Output}: partition\_key, [A / B] and the input line
        \STATE Define $N\_Partition$ as the number of nodes in the cluster
        
        \FORALL{line in STDIN}
        \STATE $products\_list$, $users\_list$ $\gets$ line.split("\texttt{\textbackslash t}")

        \IF{\textbf{size}($users\_list$) > 1}
            \STATE $node\_number \gets$ \textbf{hash}($products\_list$)

            \FOR{$i$ in $1$..$N\_Partition$}
                \STATE $partition\_key\_A \gets$ "\{$node\_number$\}-\{$i$\}"
                \STATE $partition\_key\_B \gets$ "\{$i$\}-\{$node\_number$\}"
                \STATE \textbf{print}("\{$partition\_key\_A$\}\texttt{\textbackslash t}A-\{$line$\}")
                \STATE \textbf{print}("\{$partition\_key\_B$\}\texttt{\textbackslash t}B-\{$line$\}")
            \ENDFOR
            
        \ENDIF
        
       \ENDFOR
      \end{algorithmic}
    \end{algorithm}

    \begin{algorithm}[!ht]
      \caption{Reducer2}
      \begin{algorithmic}[1]
      \STATE \textbf{Input}: input data from STDIN
      \STATE \textbf{Output}: a list of products and the list of users who bought them
      \STATE Initialize $list\_A$ as an empty list
      \STATE Initialize $list\_B$ as an empty list
      \FORALL{$in\_line$ in STDIN}
        \STATE $partition\_key\_X$, $X\_line$ $\gets$ $in\_line$.split("\texttt{\textbackslash t}")
        \STATE $X$, $line \gets$ $X\_line$.split("-")
        \IF{$X$ == "A"}
            \STATE $list\_A$.append($line$)
        \ELSE
            \STATE $list\_B$.append($line$)
        \ENDIF
      \ENDFOR
      
      \FORALL{$line1$, $line2$ pairs in $list\_A$.cartesian($list\_B$)}
        \STATE $products1$, $users1$ in $line1$
        \STATE $products2$, $users2$ in $line2$

        \STATE $common\_users \gets users1$.intersect($users2$) 
        \STATE $common\_products \gets products1$.union($products2$)

        \IF{\textbf{size}($common\_users$) $>$ 1}
        \STATE \textbf{print}(\{$common\-products$\}\texttt{\textbackslash t}\{$common\_users$]\})
        \ENDIF
      \ENDFOR
    \end{algorithmic}
    \end{algorithm}
    
    \newpage
    \item \textbf{Quarta passata}: si effettua il sort sulla base del primo elemento della lista di utenti e si eliminano le ennuple duplicate.

          \begin{algorithm}[!ht]
    \caption{Mapper3}
      \begin{algorithmic}[1]
        \STATE \textbf{Input}: CSV data from STDIN
        \STATE \textbf{Output}: all list of products and the list of users who bought them
        
        \FORALL{line in STDIN}
        \STATE $products\_list$, $users\_list$ $\gets$ line.split("\texttt{\textbackslash t}")
        \STATE \textbf{print}("\{$users\_list$[0]\}\texttt{\textbackslash t}\{$products\_list$\}\texttt{\textbackslash t}\{$users\_list$\}")
        
       \ENDFOR
      \end{algorithmic}
    \end{algorithm}
    
    \begin{algorithm}[!ht]
      \caption{Reducer3}
      \begin{algorithmic}[1]
      \STATE \textbf{Input}: input data from STDIN
      \STATE \textbf{Output}: a set of tuples containing for each $users\_list$ a set of $products$ that each group is defined as who reviewed with a score greater than or equal to 4 at least 3 products in common, sorted by the $user_id$ of the first element of the group and without duplicates.
      \STATE Initialize $dedup\_set$ as a Set

      \FORALL{$line$ in STDIN}
        \STATE $user\_0$, $products\_list$, $users\_list \gets$ $line$.split("\texttt{\textbackslash t}")
        \STATE $dedup\_set$.append("\{$products\_list$\}\texttt{\textbackslash t}\{$users\_list$\}")
      \ENDFOR
      
      \FORALL{$line$ in $dedup\_set$}
        \STATE \textbf{print}($line$)
      \ENDFOR
    \end{algorithmic}
    \end{algorithm}    
  \end{itemize}

\newpage
  \subsubsection{Hive}
    Per l'implementazione in Hive del terzo job si è pensato di suddividere l’elaborazione in varie query intermedie che facessero apprezzare meglio l’interrogazione completa, man mano salvate in tabelle. Le tabelle intermedie sono illustrate di seguito (sono state utilizzate 2 funzioni User Defined \codeinline{array_intersect} e \codeinline{array_union} prese dal repo di \href{https://github.com/klout/brickhouse/tree/8fce0ac98aef422772ac89de7a620caac47ccc9d}{brickhouse}):
\begin{itemize}
    \item \codeinline{rated\_products}: selezione delle reviews che presentano uno \codeinline{score >= 4}.
    \begin{lstlisting}[style=all, style=SQLStyle]
CREATE TABLE if not exists rated_products as
SELECT product_id, user_id
FROM reviews
WHERE score >= 4;
\end{lstlisting}
    \item \codeinline{product\_2\_users}: tabella che contiene per ciascun \codeinline{product\_id} il relativo insieme di \codeinline{product\_id} che hanno comprato quei prodotti.
    \begin{lstlisting}[style=all, style=SQLStyle]
CREATE TABLE if not exists product_2_users as
SELECT array(product_id) as products, collect_set(user_id) as users
FROM rated_products
GROUP BY product_id
HAVING count(*) > 1 AND product_id IS NOT NULL;
\end{lstlisting}
    \item \codeinline{product_2_users_1}: prodotto cartesiano tra la tabella \codeinline{product\_2\_users} con se stessa sul quale andiamo a calcolare per ciascuna coppia creata l'intersezione tra l'insieme di utenti e l’unione delle liste dei prodotti
\begin{lstlisting}[style=all, style=SQLStyle]
CREATE TABLE if not exists  product_2_users_1 as
SELECT sort_array(array_union(p1.products, p2.products)) as products, sort_array(array_intersect(p1.users, p2.users)) as users
FROM product_2_users p1 CROSS JOIN product_2_users p2
WHERE p1.products[0] < p2.products[0] AND size(array_intersect(p1.users, p2.users)) >= 2;
\end{lstlisting}
    \item \codeinline{product_2_users_2}: la stessa query di prima ma effettuata sulla tabella di \codeinline{product_2_users_1}
\begin{lstlisting}[style=all, style=SQLStyle]
CREATE TABLE if not exists  product_2_users_2 as
SELECT sort_array(array_union(p1.products, p2.products)) as products, sort_array(array_intersect(p1.users, p2.users)) as users
FROM product_2_users_1 p1 CROSS JOIN product_2_users_1 p2
WHERE p1.products[0] < p2.products[0] AND size(array_intersect(p1.users, p2.users)) >= 2;
\end{lstlisting}

    \item \codeinline{products_2_users_sorted}: la stessa tabella di prima ma senza valori duplicati e ordinata sulla base del primo elemento della lista di utenti.
\begin{lstlisting}[style=all, style=SQLStyle]
CREATE TABLE if not exists  products_2_users_sorted as
SELECT distinct(products), users
FROM product_2_users_2
ORDER BY users[0] ASC;
\end{lstlisting}
\newpage

\end{itemize}
  \subsubsection{Spark Core}
      Per l'implementazione del terzo job in spark-core sono state applicate quattro funzioni \codeinline{map()}, due \codeinline{filter()}, due \codeinline{groupByKey()} ed una \codeinline{sortBy()}.
  \begin{algorithm}
    \caption{Spark Application}
    \begin{algorithmic}[1]
        \STATE \textbf{Input}: input dataset
        \STATE \textbf{Output}: a text file containing for each group of $user\_id$ a set of $product\_id$ that each group is defined as who reviewed with a score greater than or equal to 4 at least 3 products in common, sorted by the $user_id$ of the first element of the group and without duplicates.
        
        \STATE Define $get\_users\_for\_productId()$: a function that returns the productId and userId with score >= 4 from a dataset line
        \STATE Initialize $input\_rdd$ from the input dataset
        \STATE \textbf{Filter} out the header line from the $input\_rdd$

        \STATE $productId\_for\_users\_rdd$ $\gets$ \textbf{Map} ($product\_id$, $user\_id$) from $input\_rdd$ using $get\_users\_for\_productId()$
        \STATE $products\_for\_users\_rdd$ $\gets$ \textbf{Group By Key} ($product\_id$, List of $user\_id$) from $productId\_for\_users\_rdd$
        \STATE $products\_for\_users\_rdd$ $\gets$ \textbf{Map} (Set of $product\_id$ as \textbf{$products$}, Set of $user\_id$ as \textbf{$users$}) from $products\_for\_users\_rdd$
        \FOR{\texttt{$i$ in $1$..$2$}}
            \STATE $products\_for\_users\_rdd$ $\gets$ \textbf{Cartesian Product} (($products$, $users$), ($products$, $users$))
            \STATE $products\_for\_users\_rdd \gets$ \textbf{Map} ($products$ Union as $products$, \\ $users$ Intersection as $users$)
            \STATE \textbf{Filter} out line with \textbf{size}($users$) $< 2$ from $products\_for\_users\_rdd$
        \ENDFOR
        \STATE $products\_for\_users\_rdd$ $\gets$ \textbf{Group By} ($products$) 
        \STATE $products\_for\_users\_rdd$ $\gets$ \textbf{Sort By} ($user\_id$ of the first element of the group) 

        \STATE Save as text file $products\_for\_users\_rdd$
    \end{algorithmic}
    \end{algorithm}
    
  \subsubsection{Spark SQL}
  L'implementazione del terzo job in spark-sql è stata divisa nelle seguenti operazioni:
  \begin{itemize}
    \item \codeinline{productId\_for\_users\_DF}: set di almeno 2 utenti che hanno recensito score $>=4$ quel prodotto.
  \begin{lstlisting}[style=all, style=PythonStyle]
productId_for_users_DF = input_DF.select("UserId", "ProductId").where(input_DF["Score"] >= 4) \ .groupBy("ProductId").agg(collect_set("UserId").alias("Users"))\
.where(size("Users") > 3)
  \end{lstlisting}
    \item \codeinline{products\_for\_users\_DF}: prodotto inserito nella propria lista di prodotti con solo se stesso come elemento.
  \begin{lstlisting}[style=all, style=PythonStyle]
products_for_users_DF = productId_for_users_DF.withColumn("Products", array("ProductID"))
\end{lstlisting}
  \item \codeinline{products\_for\_users\_DF}: gruppi di utenti con almeno tre prodotti in comune.
     \begin{lstlisting}[style=all, style=PythonStyle]
for i in range(2):
    products_for_users_DF = products_for_users_DF.withColumnRenamed("Users","Users1")\ .withColumnRenamed("Products", "Products1") \ .crossJoin(products_for_users_DF.withColumnRenamed("Users", "Users2")
    .withColumnRenamed("Products", "Products2")) \
    .where(F.col("Products1") < F.col("Products2"))
    products_for_users_DF = products_for_users_DF.select(array_union("Products1","Products2").alias("Products"), array_intersect("Users1", "Users2").alias("Users")).distinct() products_for_users_DF = products_for_users_DF.where(size("Products") >= 2)
\end{lstlisting}

    \item \codeinline{year\_for\_product\_2_word\_count\_DF}: calcolo delle occorrenze di ciascuna parola (con almeno 4 caratteri) di ciascun prodotto all'interno dell'anno.
     \begin{lstlisting}[style=all, style=PythonStyle]
top_10_products_for_year_with_reviews_DF = top_10_products_for_year_with_reviews_DF.withColumn("word", explode(split(top_10_products_for_year_with_reviews_DF.text, " ")))
year_for_product_2_word_count_DF = top_10_products_for_year_with_reviews_DF.groupBy( "review_year", "product_id", "word").count().withColumnRenamed("count", "word_count")\ .where("length(word) >= 4").select("review_year", "product_id", "word", "word_count")
\end{lstlisting}

    \item \codeinline{output\_DF}: rimozione dei duplicati e ordinamento in base allo UserId del primo elemento del gruppo.
     \begin{lstlisting}[style=all, style=PythonStyle]
output_DF = products_for_users_DF.groupBy("Products")\
.orderBy(products_for_users_DF["Users"][0])
\end{lstlisting}
  \end{itemize}

\newpage
\section{Risultati e tempi di esecuzione}
In questa sezione sono riportati gli output e le prestazioni ottenute dall'esecuzione delle varie implementazioni. Al fine di effettuare vari test comparativi prestazionali sulle diverse implementazioni utilizzate per i job illustrati verranno utilizzate diversi dataset con numero di ennuple crescenti ottenuti effettuando downsampling e oversampling dalla versione del dataset ottenuta post-processing, il dettaglio verrà approfondito nella sezione apposita. Per quanto riguarda gli output dei job invece, è stata utilizzata la versione del dataset ottenuta post-processing con percentuale del dataset pari al 100\%, ovvero con numero di ennuple originali.

\subsection{Output}
Di seguito sono riportate le prime 10 linee ottenuta dall'esecuzione delle varie implementazioni di tutti i job presentati in \ref{section:1}:
\begin{itemize}
    \item \textbf{Job 1}:
    \begin{table}[!ht]
      \centering
      \begin{minipage}{.5\textwidth}
        \centering
        \caption{Map-Reduce}
        \begin{tabular}{cccc}
          \toprule
          Year & ProductId & Word & Count \\
          \midrule
          2012 & B001VJ0B0I & food & 1563 \\
          2012 & B001VJ0B0I & this & 1074 \\
          2012 & B001VJ0B0I & that & 908 \\
          2012 & B001VJ0B0I & dogs & 631 \\
          2012 & B001VJ0B0I & with & 609 \\
          2012 & B0041NYV8E & ginger & 1446 \\
          2012 & B0041NYV8E & this & 1086 \\
          2012 & B0041NYV8E & that & 660 \\
          2012 & B0041NYV8E & drink & 623 \\
          2012 & B0041NYV8E & lemon & 584 \\
          \bottomrule
        \end{tabular}
      \end{minipage}%
      \begin{minipage}{.5\textwidth}
        \centering
        \caption{Hive}
        \begin{tabular}{cccc}
          \toprule
          Year & ProductId & Word & Count \\
          \midrule
          1999 & 0006641040 & this & 5 \\
          1999 & 0006641040 & when & 4 \\
          1999 & 0006641040 & book & 3 \\
          1999 & 0006641040 & along & 2 \\
          1999 & 0006641040 & recite & 2 \\
          1999 & B00004CI84 & rumplestiskin & 1 \\
          1999 & B00004CI84 & with & 1 \\
          1999 & B00004CI84 & rumbles & 1 \\
          1999 & B00004CI84 & prime & 1 \\
          1999 & B00004CI84 & point & 1 \\
          \bottomrule
        \end{tabular}
      \end{minipage}
    \end{table}
    
    \begin{table}[!ht]
      \centering
      \begin{minipage}{.5\textwidth}
        \centering
        \caption{Spark Core}
        \begin{tabular}{cccc}
          \toprule
          Year & ProductId & Word & Count \\
          \midrule
          1999 & 0006641040 & this & 5 \\
          1999 & 0006641040 & when & 4 \\
          1999 & 0006641040 & book & 3 \\
          1999 & 0006641040 & books & 2 \\
          1999 & 0006641040 & along & 2 \\
          1999 & B00004CI84 & twist & 1 \\
          1999 & B00004CI84 & rumplestiskin & 1 \\
          1999 & B00004CI84 & captured & 1 \\
          1999 & B00004CI84 & film & 1 \\
          1999 & B00004CI84 & starring & 1 \\
          \bottomrule
        \end{tabular}
      \end{minipage}%
      \begin{minipage}{.5\textwidth}
        \centering
        \caption{Spark SQL}
        \begin{tabular}{cccc}
          \toprule
          Year & ProductId & Word & Count \\
          \midrule
          1999 & 0006641040 & this & 5 \\
          1999 & 0006641040 & when & 4 \\
          1999 & 0006641040 & book & 3 \\
          1999 & 0006641040 & along & 2 \\
          1999 & 0006641040 & recite & 2 \\
          2000 & B00002N8SM & flies & 3 \\
          2000 & B00002N8SM & after & 2 \\
          2000 & B00002N8SM & quot & 2 \\
          2000 & B00002N8SM & this & 2 \\
          2000 & B00002N8SM & bought & 1 \\
          \bottomrule
        \end{tabular}
      \end{minipage}
    \end{table}
    \newpage
    
    \item \textbf{Job 2}:
    \begin{table}[!ht]
      \centering
      \begin{minipage}{.5\textwidth}
        \centering
        \caption{MapReduce}
        \begin{tabular}{cc}
          \toprule
          UserId & Appreciation \\
          \midrule
            A1EUO0BU72JR7T & 1.0 \\
            A1EUMUYR3HUQPA & 1.0 \\
            A1EUMJC8VE44QL & 1.0 \\
            A1EUM0NGV6SRSQ & 1.0 \\
            A1EULW892J1V5U & 1.0 \\
            A1EUKV9UEKWP0K & 1.0 \\
            A1EUKT4PPNJUEI & 1.0 \\
            A1EUI8B624BU88 & 1.0 \\
            A1EUHSLJ41U830 & 1.0 \\
            A1EUH5T3M85UF1 & 1.0 \\
          \bottomrule
        \end{tabular}
      \end{minipage}%
      \begin{minipage}{.5\textwidth}
        \centering
        \caption{Hive}
        \begin{tabular}{cc}
          \toprule
          UserId & Appreciation \\
          \midrule
              AZZU4D6TZ2L6J & 1.0 \\
              AZZRFMUO60L7J & 1.0 \\
              AZZMDW27MUJR6 & 1.0 \\
              AZZFJQFHITBZ5 & 1.0 \\
              AZZ1VDC2O1VUN & 1.0 \\
              AZYW6LUFW2RFN & 1.0 \\
              AZYVD25M8Y4A2 & 1.0 \\
              AZYTCGDO3R4ZU & 1.0 \\
              AZYIAIP0PE9SE & 1.0 \\
              AZYEQBVY8AVVW & 1.0 \\
          \bottomrule
        \end{tabular}
      \end{minipage}
    \end{table}
    \begin{table}[!ht]
      \centering
      \begin{minipage}{.5\textwidth}
        \centering
        \caption{Spark Core}
        \begin{tabular}{cc}
          \toprule
          UserId & Appreciation \\
          \midrule
              A2TQ5EE53V34SX & 1.0 \\
              A1EAX5HVYV96EF & 1.0 \\
              A1Y64EPRAH20CX & 1.0 \\
              AI1LL9G6L41KC & 1.0 \\
              A36YQZO4XQ6KOD & 1.0 \\
              A22JSUHQXNFOXT & 1.0 \\
              A32KHOAU900IH0 & 1.0 \\
              A2A6AMJRB2M2PJ & 1.0 \\
              A1L4FDB71OYY7Q & 1.0 \\
              A3CERM4YQI6DWD & 1.0 \\
          \bottomrule
        \end{tabular}
      \end{minipage}%
      \begin{minipage}{.5\textwidth}
        \centering
        \caption{Spark SQL}
        \begin{tabular}{cc}
          \toprule
          UserId & Appreciation \\
          \midrule
              AVCA516CFZ9HF & 1.0 \\
              A3ARZW7S96SESC & 1.0 \\
              A2CJGVAHVZGJIP & 1.0 \\
              A7TCUZ9EVNRS0 & 1.0 \\
              A3NIJP0MX60FW3 & 1.0 \\
              A1TL8ME5TPUM2Y & 1.0 \\
              A2N2UYF083BZW0 & 1.0 \\
              AMDPYQR7QVOJQ & 1.0 \\
              A30BNNI1U9FCIV & 1.0 \\
              A1M5YCQ2BPI6ES & 1.0 \\
          \bottomrule
        \end{tabular}
      \end{minipage}
    \end{table}
    \newpage
    
    \item \textbf{Job 3}:
    \begin{table}[!ht]
        \centering
        \caption{Map-Reduce}
        \begin{tabular}{cc}
          \toprule
          Common products &  Group of like-minded users \\
          \midrule
          B005HG9ESG,B005HG9ET0,B005HG9ERW & \hyperlink{Group1}{Group1} \\
          B007KIOBMS,B003PFPFIE,B000MTM0WK & \hyperlink{Group2}{Group2} \\
          B000MTM0WK,B003PFPFIE,B007KIOBMS & \hyperlink{Group2}{Group2} \\
          B002R8UANK,B002R8J7YS,B002R8SLUY & \hyperlink{Group3}{Group3} \\
          B005DGI1VG,B005DGI1IY,B005DGI242,B005EXGE5I & \hyperlink{Group4}{Group4} \\
          B005DGI1IY,B005DGI1VG,B005DGI1PW,B005DGI2II & \hyperlink{Group4}{Group4} \\
          B005DGI242,B005EXGE5I,B005DGI2II & \hyperlink{Group4}{Group4} \\
          B005DGI1VG,B005DGI242,B005EXGE5I,B005DGI2II & \hyperlink{Group4}{Group4} \\
          B005DGI1VG,B005DGI242,B005EXGE5I & \hyperlink{Group4}{Group4} \\
          B005DGI242,B005DGI1PW,B005EXGE5I,B005DGI2II & \hyperlink{Group4}{Group4} \\
          \bottomrule
        \end{tabular}
    \end{table}
    
    \begin{table}[!ht]
    \centering
    \caption{Hive}
    \begin{tabular}{cc}
      \toprule
          Common products &  Group of like-minded users \\
          \midrule
          B005HG9ESG,B005HG9ET0,B005HG9ERW & \hyperlink{Group1}{Group1} \\
          B007KIOBMS,B003PFPFIE,B000MTM0WK & \hyperlink{Group2}{Group2} \\
          B000MTM0WK,B003PFPFIE,B007KIOBMS & \hyperlink{Group2}{Group2} \\
          B002R8UANK,B002R8J7YS,B002R8SLUY & \hyperlink{Group3}{Group3} \\
          B005DGI1VG,B005DGI1IY,B005DGI242,B005EXGE5I & \hyperlink{Group4}{Group4} \\
          B005DGI1IY,B005DGI1VG,B005DGI1PW,B005DGI2II & \hyperlink{Group4}{Group4} \\
          B005DGI242,B005EXGE5I,B005DGI2II & \hyperlink{Group4}{Group4} \\
          B005DGI1VG,B005DGI242,B005EXGE5I,B005DGI2II & \hyperlink{Group4}{Group4} \\
          B005DGI1VG,B005DGI242,B005EXGE5I & \hyperlink{Group4}{Group4} \\
          B005DGI242,B005DGI1PW,B005EXGE5I,B005DGI2II & \hyperlink{Group4}{Group4} \\
          \bottomrule
    \end{tabular}
  \end{table}

    \begin{table}[!h]
    \centering
    \caption{Spark-Core}
    \begin{tabular}{cc}
      \toprule
          Common products &  Group of like-minded users \\
          \midrule
          N/A\textbf{*}  & N/A\textbf{*} \\
          \bottomrule
    \end{tabular}
  \end{table}

    \begin{table}[!h]
    \centering
    \caption{Spark-SQL}
    \begin{tabular}{cc}
      \toprule
          Common products &  Group of like-minded users \\
          \midrule
          N/A\textbf{*}  & N/A\textbf{*} \\
          \bottomrule
    \end{tabular}
  \end{table}

\textbf{*} I valori N/A\textbf{*} di output dalle implementazioni di Spark-Core e Spark-SQL sono stati osservati, ma per correttezza non aggiunti poichè si è notato che queste implementazioni per il terzo job una volta introdotto il codice per rimuovere i duplicati e l'ordinamento dei risultati sulla base del primo elemento del gruppo creava un outOfMemory di spark sia in locale che in cluster, rimuovendo questo pezzo di codice è possibile osservare l'output, giudicato corretto fino a tale punto.
\newpage    

\end{itemize}

\subsection{Configurazione Software/Hardware}
La configurazione del sistema in locale con il quale sono stati effettuati i test in locale è, CPU 11th Gen Intel(R) Core(TM) i5-1135G7 @ 2.40GHz 2.42 GHz, 16GB ram. Mentre la configurazione del cluster ERM utilizzato per eseguire i test su AWS è: 1 master, 4 slave m5.xlarge con 4 vCore, 16GB ram.
\subsection{Tempi di esecuzione}
Per valutare le performance (temporali) ed effettuare una comparazione tra MapReduce, Hive, Spark-Core e Spark-SQL, ciascuna implementazione di ciascun job è stata eseguita in locale e su cluster utilizzando dimensioni crescenti del dataset, rispettivamente \textbf{100\%}, \textbf{200\%}, \textbf{250\%}, \textbf{500\%}, \textbf{750\%} e \textbf{1000\%} del dataset più uno al \textbf{50\%} usato solo per i test del terzo job per questioni di tempo. Per effettuare queste valutazioni è stata calcolata la media dei tempi d'esecuzione trovati in ciascuna esecuzione su un totale di 10 esecuzioni.

\subsubsection{Job 1 - locale}
\begin{figure}[!ht]
    \centering
    \includegraphics{barplot_times_1_job_local}
    \caption{BarPlot Performance Comparison - 1st job}
\end{figure}

\begin{table}[!ht]
  \centering
  \caption{Table Performance Comparison local (in seconds) - 1st job}
  \label{tab:performance}
  \begin{tabular}{c|c|c|c|c}
    \toprule
    Dataset Percentage & MR & Hive & Spark-Core & Spark-SQL \\
    \midrule
    100\% & 36.894 & 115.154 & 55.904 & \textbf{36.393} \\
    200\% & 46.683 & 100.126 & 57.226 & \textbf{43.824} \\
    250\% & 55.944 & 106.370 & 58.050 & \textbf{49.136} \\
    500\% & 92.598 & 167.288 & 80.570 & \textbf{63.736} \\
    750\% & 101.637 & 209.778 & 99.100 & \textbf{89.417} \\
    1000\% & 188.920 & 278.802 & 122.451 & \textbf{105.177} \\
    \bottomrule
  \end{tabular}
\end{table}
\newpage
\subsubsection{Job 2 - locale}
\begin{figure}[!ht]
    \centering
    \includegraphics{barplot_times_2_job_local}
    \caption{BarPlot Performance Comparison - 2nd job}
\end{figure}

\begin{table}[!ht]
  \centering
  \caption{Table Performance Comparison local (in seconds) - 2nd job}
  \label{tab:performance}
  \begin{tabular}{c|c|c|c|c}
    \toprule
    Dataset Percentage & MR & Hive & Spark-Core & Spark-SQL \\
    \midrule
    100\% & 38.527 & 52.363 & 35.262 & \textbf{34.944} \\
    200\% & 39.140 & 52.086 & \textbf{36.502} & 46.677 \\
    250\% & 40.440 & 52.422 & \textbf{38.021} & 48.148 \\
    500\% & 50.693 & 61.574 & \textbf{46.188} & 63.883 \\
    750\% & 64.970 & 75.902 & \textbf{54.407} & 76.778 \\
    1000\% & 74.446 & 84.166 & \textbf{68.106} & 92.926 \\
    \bottomrule
  \end{tabular}
\end{table}
\newpage

\subsubsection{Job 3 - locale}
\begin{figure}[!ht]
    \centering
    \includegraphics{barplot_times_3_job_local.png}
    \caption{BarPlot Performance Comparison - 3nd job}
\end{figure}

\begin{table}[!ht]
  \centering
  \caption{Table Performance Comparison local (in seconds) - 3nd job}
  \label{tab:performance}
  \begin{tabular}{c|c|c|c|c}
    \toprule
    Dataset Percentage & MR & Hive & Spark-Core & Spark-SQL \\
    \midrule
    50\% & 349.182 & \textbf{175.801} & N/A & N/A \\
    100\% & 3780.574 & \textbf{1860.617} & N/A & N/A \\
    200\% & N/A & N/A & N/A & N/A \\
    250\% & N/A & N/A & N/A & N/A \\
    500\% & N/A & N/A & N/A & N/A \\
    750\% & N/A & N/A & N/A & N/A \\
    1000\% & N/A & N/A & N/A & N/A \\
    \bottomrule
  \end{tabular}
\end{table}
\newpage


\subsubsection{Job 1 - cluster AWS}

\begin{figure}[!ht]
    \centering
    \includegraphics{barplot_times_1_job_AWS}
    \caption{BarPlot Performance Comparison - 1nd job}
\end{figure}

\begin{table}[!ht]
  \centering
  \caption{Table Performance Comparison AWS (in seconds) - 1st job}
  \label{tab:performance}
  \begin{tabular}{c|c|c|c|c}
    \toprule
    Dataset Percentage & MR & Hive & Spark-Core & Spark-SQL \\
    \midrule
     100\% & 31.621 & 66.882 & 36.359 & \textbf{21.566} \\
    200\% & 39.531 & 58.011 & 37.894 & \textbf{26.088} \\
    250\% & 47.489 & 61.729 & 37.770 & \textbf{28.727} \\
    500\% & 78.527 & 96.663 & 53.047 & \textbf{37.768} \\
    750\% & 85.653 & 121.778 & 64.961 & \textbf{53.008} \\
    1000\% & 162.075 & 162.785 & 79.716 & \textbf{61.721} \\
    \bottomrule
  \end{tabular}
\end{table}
\newpage

  
\subsubsection{Job 2 - cluster AWS}

\begin{figure}[!ht]
    \centering
    \includegraphics{barplot_times_2_job_AWS}
    \caption{BarPlot Performance Comparison - 2nd job}
\end{figure}

\begin{table}[!ht]
  \centering
  \caption{Table Performance Comparison AWS (in seconds) - 2nd job}
  \label{tab:performance}
  \begin{tabular}{c|c|c|c|c}
    \toprule
    Dataset Percentage & MR & Hive & Spark-Core & Spark-SQL \\
    \midrule
    100\% & 52.161 & 34.717 & 26.610 & \textbf{21.413} \\
    200\% & 52.613 & 34.050 & \textbf{27.668} & 28.966 \\
    250\% & 54.542 & 34.695 & \textbf{29.224} & 29.855 \\
    500\% & 68.595 & 40.200 & \textbf{35.090} & 39.568 \\
    750\% & 87.600 & 50.005 & \textbf{41.609} & 47.968 \\
    1000\% & 98.615 & 55.541 & \textbf{51.963} & 57.607 \\
    \bottomrule
  \end{tabular}
\end{table}
\newpage

\subsubsection{Job 3 - cluster AWS}

\begin{figure}[!ht]
    \centering
    \includegraphics{barplot_times_3_job_AWS}
    \caption{BarPlot Performance Comparison - 3nd job}
\end{figure}

\begin{table}[!ht]
  \centering
  \caption{Table Performance Comparison AWS (in seconds) - 3nd job}
  \label{tab:performance}
  \begin{tabular}{c|c|c|c|c}
    \toprule
    Dataset Percentage & MR & Hive & Spark-Core & Spark-SQL \\
    \midrule
    50\% & N/A & \textbf{2200.71} & N/A & N/A \\
    100\% & N/A & \textbf{2400.39} & N/A & N/A \\
    200\% & N/A & N/A & N/A & N/A \\
    250\% & N/A & N/A & N/A & N/A \\
    500\% & N/A & N/A & N/A & N/A \\
    750\% & N/A & N/A & N/A & N/A \\
    1000\% & N/A & N/A & N/A & N/A \\
    \bottomrule
  \end{tabular}
\end{table}
\newpage

\subsubsection{Grafici a confronto sui job in locale e AWS}
Di seguito sono riportati i grafici per confrontare i tempi di esecuzione sui diversi job, in locale e in cluster AWS, dove, per la mancanza di dati non è stato preso in considerazione il terzo job nè in locale nè in cluster.
\begin{figure}[!ht]
    \centering
    \includegraphics[height=0.6\textwidth]{compare_barplt_MR.png}
\end{figure}

\begin{figure}[!ht]
    \centering
    \includegraphics[height=0.6\textwidth]{compare_barplt_Hive.png}
\end{figure}
\newpage
\begin{figure}[!ht]
    \centering
    \includegraphics[height=0.6\textwidth]{compare_barplt_SparkCore.png}
\end{figure}

\begin{figure}[!ht]
    \centering
    \includegraphics[height=0.6\textwidth]{compare_barplt_SparkSQL.png}
\end{figure}
\newpage

\subsubsection{Gruppi Job 3}
\begin{itemize}
\item \textbf{\hypertarget{Group1}{Group1}}: \\
    \#oc-R11D9D7SHXIJB9, \#oc-R11O5J5ZVQE25C,\#oc-R12MGTQS5KZZRV, \\ \#oc-R14ZSRYW2YB41B, \#oc-R155JB2SA58E17, \#oc-R162D7S0A880MV, \\ \#oc-R163CP16SRRI50, \#oc-R19EJ3VEA88T6O, \#oc-R1CF9LIM90SAB6, \\ \#oc-R1GKUU1PTIDIZ7, \#oc-R1GSBW9QIVY489, \#oc-R1K4OCJ8HEIEDY, \\ \#oc-R1QHGBT11WAS7G, \#oc-R1U8FAII0QYOT3, \#oc-R1UO6NAAGVBW7Z, \\ \#oc-R1VRD09DW4H2HI, \#oc-R26DKYCO954ZWP, \#oc-R28I1AL1ZAZLXL, \\ \#oc-R2B86BJE5FNKXX, \#oc-R2HWL8UHAIMFRS, \#oc-R2K9AJ2LWO7ZUJ, \\ \#oc-R2M17G7NB9RAIV, \#oc-R2MG5C3DMRU8Q0, \#oc-R2R45EEG606NCJ, \\ \#oc-R2W0C6DARSLJ0S, \#oc-R2W66Y63G88976, \#oc-R2XZVYL146WRFL, \\ \#oc-R2YPVWM76O2TFX, \#oc-R31AI08Q9HLVF1, \#oc-R34PQ2ORKQ6WCD, \\ \#oc-R3DERHJ8UWPZZ, \#oc-R3EDGA2893DM4Y, \#oc-R3F0UDHOQC1RNU, \\ \#oc-R3OS88C8I7GSS5, \#oc-R3RMG4F3JAN4CE, \#oc-R3RQNMHS7481DE, \\ \#oc-R3SRKE3YQ2BNES, \#oc-R3TXZAQ0JD85LR, \#oc-R3TZTZQ7UHNM8Y, \\ \#oc-RKU1BA6XFA3Q4, \#oc-RMBODWNVK1H1D, \#oc-RODMQLZAXFCUG, \\ \#oc-RS2VFRNYYRVUG, \#oc-RSZMLJGJRON5W, \#oc-RUCLMJ3IUSWPC, \\ \#oc-RXCJ97CMQTXVA, \#oc-RXFX0AFFBDSJN, A10LDAXFO052F7, \\ A119SZPTE05762, A11I1I9QLMAM1A, A11OBEH2ZCVAHX, \\ A11OTLEDSW8ZXD, A14EF1PPKMSEPU, A14X244VGHWPSX, \\ A161BVWC65FWE5, A174GR3NPUAPPN, A18OUQEK7IZ2F2, \\ A194K60CWZ371J, A19EKT8H85AKO5, A19FRW264WZTGP, \\ A19XMHRB3G4DIR, A1A535TCGNMVPI, A1AEPMPA12GUJ7, \\ A1BK60GZ4QME6I, A1E50L7PCVXLN4, A1E5WXNTI7EO2A, \\ A1EVV74UQYVKRY, A1F1A0QQP2XVH5, A1F7YU6O5RU432, \\ A1HRYC60VTMYC0, A1ITRGMT80D5TK, A1JMR1N9NBYJ1X, \\ A1JV4QKTEB7QBL, A1KEK09ZA6J9P8, A1KXJCXS6HFRQZ, \\ A1L1S42BOGPF96, A1LA4K5JF78BER, A1MZL91Z44RN06,\\ A1NOV41485TE0D, A1ODOGXEYECQQ8,  A1OHGX0LKS6QLP,\\ A1ONZ8JRPLBNUI, A1OWRLSD7LTSQ9, A1P2XYD265YE21,\\ A1PI8VBCXXSGC7, A1PQTLSDHCK8II, A1QBOC76MIOJYP,\\ A1R1BFJCMWX0Y3, A1UHKF1UQ3EGW4, A1UMSB7LAW0RIR,\\ A1UQBFCERIP7VJ, A1VF5LN6SHFVFJ, A1VYFEJM12ZP11,\\ A1W2EGUPW8OYH4, A1W415JP5WEAJK, A1W60EW7Q6A75C,\\ A1WKQ94M45D8MG, A1Y5ERIKQBEMSP, A1YJMG0QJXZLD4,\\ A1ZCSSCAGBCD49, A1ZENB34HH2EIM, A1ZPY91VE3IDN1,\\ A1ZU55TM45Y2R8, A2068BC3ZXAVJQ, A20V7N5A22F4BM,\\ A215WH6RUDUCMP, A217CC8F7N5717, A219J1R4Z4BF7R,\\ A226VGZWOEBPGL, A22CW0ZHY3NJH8, A22KL4WOK6GTW2,\\ A235UFZGCFN3J5, A23RZIU0N8K2KR, A244CRJ2QSVLZ4,\\ A24U6WGBZ4P74W, A2561PYW9TTMYD, A25C5MVVCIYT5D,\\ A25HYPL2XKQPZB, A25KVM6GJBLISZ, A267FU71Z01CIH,\\ A26LD9FQHTM8ZH, A26TYDQ2BFD4EG, A277TO3PKKNYDH,\\ A27CK29BVQNGD9, A27O5ZUFAEWT9L, A294SHLWPSG1BP,\\ A29M09QBG9TZLP, A2AWVROFGSZU4E, A2BE3MTUT0YBF0,\\ A2CRIQOX1SYA3I, A2D7B5I7ZQ51XL, A2DPYMNI2HCIOI,\\ A2E7RX6AFUDQEX, A2F09EWKV3MTO2, A2GPJR489OIH42,\\ A2HTPS0JV3Q8ZD, A2IMLOJ2Z2QEU2, A2JAEZ0FMAMJVW,\\ A2L0WJMOT484GM, A2L4ZGN7GZJ95T, A2L6BT1PVV9YN0,\\ A2LRNLAV0ZIL4U, A2MGNLN42OKJHC, A2MJ8OL2FYN7CW,\\ A2OCDK0BOW6UCY, A2OR4QUQSUMOW7, A2PD27UKAD3Q00,\\ A2PSC7LUNIDEAH, A2R1HUYHXV7H18, A2RHV42BTJSVON,\\ A2S26YGSVXBCFL, A2TBAUW2W7J538, A2UEB48LAWFUCW,\\ A2UGR7VULJBQ2N, A2UW9WI22QKMZE, A2V0I904FH7ABY,\\ A2V92F5R7MLCVI, A2W9I628I6SE1U, A2WB4OWBUH2VQX,\\ A2WN1QF8GSVHYV, A2WW57XX2UVLM6, A2X3L31KCXBHCL,\\ A2XIOXRRYX0KZY, A2XSY4L7GDHV4W, A2YB7DLC3FOR7W,\\ A2YKWYC3WQJX5J, A2YQ2ZI65F37N8, A2ZPZM9RE08JXF,\\ A3094EPI56GKZ6, A30LSAC7UMZDWS, A30Z3BVRNUM3NW,\\ A319Y83RT0MRVR, A31BD4RXCON7QO, A31COKWABW2WH2,\\ A31RULW0KNYJ5H, A32EOHLFZYXJEP, A32XGYDA14KT13,\\ A339F4I8GBN3H3, A33Y8C4818EJL0, A353U0L2HAMSHV,\\ A36K2N527TXXJN, A36MP37DITBU6F, A37DXIENIDHTVY,\\ A37FFWZUGO8L7W, A37OYVYHR5U4NU, A38U7MGM8XLVI2,\\ A39WWMBA0299ZF, A3A7Y3TSPPZU9T, A3AQO91DYWF8R0,\\ A3BAE79NXFDXGV, A3BFDEBT5IV4UN, A3BKNXX8QFIXIV,\\ A3CA3RWZYJDWXE, A3DOPYDOS49I3T, A3DRSOGQJRX10,\\ A3DZFEICHK5LF2, A3EFSLEMHNPP6A, A3EOVXI1VZIHUQ,\\ A3F3B6HY9RJI04, A3FMPT5IH0CJ50, A3GRN6J64F2C3X,\\ A3H7NPNDMGLOU4, A3I1BJIFFM4S21, A3K91X9X2ARDOK,\\ A3KDO3XV0MK1GX, A3LGT6UZL99IW1, A3M06TE1J42O3T,\\ A3MUSWDCTZINQZ, A3N4CUW4UYC9I3, A3NCIN6TNL0MGA,\\ A3NM1MT3Q2FHXV, A3NOBH42C7UI5M, A3OJX18B60PJR9,\\ A3Q1EDCBV2KU8D, A3QNQQKJTL76H0, A3QS4WWC1LCA6H,\\ A3QTW5LIX5SB6F, A3QVAKVRAH657N, A3RR2P5IS3DGPR,\\ A3SMR4HRFJARSC, A3SMV35531YME, A3VBXQKRM7A4JR,\\ A3VI2VETB90ZG5, A3VJ27010XUWTF, A44OY8EFDM4IP,\\ A4D5B7Q8A7PA6, A55LS2HWPQB0Q, A91TB0WX94MHP,\\ A9KLAL1CXZ0W5, AC0HPFQVBZVGY, AEJAGHLC675A7,\\ AEL6CQNQXONBX, AG6JWI77UMETS, AGEKVD8JPZQMT,\\ AHUT55E980RDR, AIEEK7AHXKZCC, AJ2FDNZ3COBFN,\\ AJRFZ0VZ0LD26, AKJHHD5VEH7VG, ALNFHVS3SC4FV,\\ ALSAOZ1V546VT, AN0U0GNJJPEUR, ANTN61S4L7WG9,\\ APDPA11IZPYLN, APP35M28G2U51, AQ6AGWKWMTY8H,\\ AQM0K7MBBT4AY, ARBKYIVNYWK3C, AS44QEHT3KSPK,\\ ASB4QD6YZJ7EX, ASJ0MKRFZC47B, ATANE2SC44592,\\ ATLO3YXU2BC16, AVJJ2D4G5I0Z4, AW7BIYHXUIZ62,\\ AY6A8KPYCE6B0, AYB4ELCS5AM8P, AYNAH993VDECT,\\ AYNRALJ4X1COS.
    
    \item \textbf{\hypertarget{Group2}{Group2}}: \\
    A015565634RZNSDLJBE5M, A100WO06OQR8BQ, A153Z14M79BKXB,\\ A16Q6SQ525EWQ5, A1MREIREZPBQEM, A1U8Y12CEPCBLV,\\ A23YGAIIIVTXEJ, A250AXLRBVYKB4, A2BCPW3PAXDQO8,\\ A2D0XVKNI8BN4A, A2DSD897956MZ7, A2QLCPGTTTIXIG,\\ A2WHNMZDUT6FJD, A2YHXAZLCLDT8D, A2ZRHMDYE6GM99,\\ A37MGGH3IQN5N2, A3HEXX70YUO8GG, A3LIW8GEAJHTYS,\\ A3MAZ108GUY8E, A3PVQGDJCOUZD1, A3S8ZXHA6XED0R,\\ A5IY506P3XG3N, AC1Z18O8ZKKYF, AC6O7RG7C7JHL,\\ AF5L8Y1LNS5G2, AFRZ1104BVZI3, AGX98VPQIR6EA,\\ ALJK8HT7T6YBT, ALL9XFM0Q1N4E

    \item \textbf{\hypertarget{Group3}{Group3}}: \\
    A09539661HB8JHRFVRDC, A110S3BQL5I1FE, A135LSE7SZE4BS,\\ A14G9DGRMBDV2U, A16AXQ11SZA8SQ, A16BVTW5Z12JDN,\\ A18TOZ8633JCXK, A19XFJFZ7PRSD2, A1A8JORZKBNRVE,\\ A1ABGR2LODX1W4, A1AJPYEH052IAT, A1BKNQFU3DGXBT,\\ A1BO2VV20MJDB3, A1E7PX99QBET2V, A1GXBRQK3COZO6,\\ A1GYCLXKB0Q26G, A1IXHI7IT67YLW, A1MG108H5E3JCC,\\ A1OFIQMDMBCY0W, A1PGF9SG78OEOT, A1QLQCJVGCQPP0,\\ A1RC1HERCUGHKC, A1RFGLD33EFO1M, A1RJVX3BGIG7W0,\\ A1S4G32UCKL9NB, A1TM3BTAYL40TB, A1UL64AZIAO2EC,\\ A1YCNM79B8EJSL, A209W98KSM74EJ, A21YKVC1JLU3LQ,\\ A24M48TURQP8QJ, A26LD9FQHTM8ZH, A26YYXWNA5AKKK,\\ A276BKCK4ZE7EW, A27JW36LL692QF, A29C2G5YUPYYVM,\\ A29DK9VBXKNB6Y, A29RN08WUB8WF2, A2CJVQQT09MRDX,\\ A2DNP1OYGWWXMT, A2FB95H8NZ27Z4, A2FU4R0UIB68BZ,\\ A2P6QIM8V4Q35E, A2PCIWAVOE2ZK5, A2PS60GWNK5VJ0,\\ A2PY7DQZTIWQOX, A2R94GSBDXG9SU, A2RTA0NCGYGCFC,\\ A2VKGUC5FN5QCY, A2YYSK2Q391XVH, A32487P9JX4520,\\ A324A43N1MU8PY, A32QYMDC6BROI3, A35TIC6B8MVKWG,\\ A370R2HK94LR43, A375R453E582H3, A37B5UD233DOF,\\ A37D9XMXYZ367Z, A38QBGUCE7N5F5, A39JF4NO5TYNNI,\\ A3C00N4UHA5XYX, A3C9NJBL1QQN3T, A3CAZVFHLRQEXQ,\\ A3G0FL3LSYL3FI, A3JH5ZLRZT1GKK, A3RFNCMOL3FSE8,\\ A3TQDIJYJPE98A, A3URGVZA532GOF, A3VX6X9LSLN6HW,\\ A50C9OS3H26GX, A7WLAB8NXTR8O, A8TQI80KFZTSD,\\ ACAIEIV03NBHY, AHZR11DB40FG4, AM6AIM58WJ3TD,\\ ATP5JJ1PE30R5, ATWCN4DHSVD2S, AUIFSR1S6JB50,\\ AVEHS8PMSVYJ1, AVI0L11EI0LBB, AXB8OD9ODILFU,\\ AZHMUHW3FAOC5.

    \item \textbf{\hypertarget{Group4}{Group4}}: \\
    A1001WMV1CL0XH, A275DMHZMRS9ZN, A2NYSPU3ILRJGP,\\ A2Q4XCKPUKKHX3, A3J2ECWH1TEQXW
\end{itemize}

\end{document}